\documentclass[]{article}

\usepackage[parfill]{parskip}

\usepackage{hyperref}

\usepackage{amssymb}

\usepackage{listings}

\usepackage{fixltx2e}

\usepackage[T1]{fontenc}

\usepackage[T1]{fontenc}

\usepackage{color}
 
\definecolor{codegreen}{rgb}{0,0.6,0}
\definecolor{codegray}{rgb}{0.5,0.5,0.5}
\definecolor{codepurple}{rgb}{0.58,0,0.82}
\definecolor{backcolour}{rgb}{0.95,0.95,0.92}
 
\lstdefinestyle{mystyle}{
    backgroundcolor=\color{backcolour},   
    commentstyle=\color{codegreen},
    keywordstyle=\color{magenta},
    numberstyle=\tiny\color{codegray},
    stringstyle=\color{codepurple},
    basicstyle=\footnotesize,
    breakatwhitespace=false,         
    breaklines=true,                 
    captionpos=b,                    
    keepspaces=true,                 
    numbers=left,                    
    numbersep=5pt,                  
    showspaces=false,                
    showstringspaces=false,
    showtabs=false,                  
    tabsize=2
}
 
\lstset{style=mystyle}

%\sectionfont{\fontsize{10}{10}\selectfont}


\begin{document}


\author{
		Harman Kumar\\
		2013CS10224		
		}

\title{Fredman-Tarjan.cpp Documentation}
\maketitle


\section{The Union Find:}

\begin{flushleft}

The class UF is an implementation of the Union Find data structure. Union by rank and path compression heuristics were added to make the implementation efficient.
	\lstinputlisting[language=c++]{UF.h}	
	\vspace{10px}	

\end{flushleft} 


\section{FibonacciHeap.h}
\begin{flushleft}
	In order to implement the Fredman-Tarjan algorithm, the fibonacci heap data structure was implemented.\\
	Following is the structure of a node of the fibonacci heap:
	
	\lstinputlisting[language=c++]{Fib_Node.h}	

	Following is the structure of the fibonacci heap (only user accesible functions are mentioned here):	

	\lstinputlisting[language=c++]{Fib.h}	
	\vspace{10px}	
	
\end{flushleft} 

\section{fredman\_tarjan(UF disjoint\_set)}
\begin{flushleft}
\begin{enumerate}

	\item One iteration of the fredman-tarjan algorithm takes as input the set of "super-vertices" on which the MST has to be calculated.\\
	
	\item It performs an iteration of the algorithm on the vertices and stores the edges that have been added to the MST.
	
	\item Finally a contraction is performed on the set of edges and vertices. 

\end{enumerate}	
\end{flushleft}

\section{I/O specification}

	The graph is read from \textbf{RandomGraph.txt}, stored in the form of an adjacency list and the MST is written in \textbf{tarjan\_time.txt} 

\section{Requirements and Execution}

The system must have c++11 and boost libraries on the system.

To execute the code:\\ 
g++ -o0 -I <path to boost libraries> -std=c++11 fredman-tarjan\_fibonacci\_heap.cpp -o tarjan

\end{document}