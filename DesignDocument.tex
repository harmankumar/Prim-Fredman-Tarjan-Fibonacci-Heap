\documentclass[]{article}

\usepackage[parfill]{parskip}

\usepackage{hyperref}

\usepackage{amssymb}

\usepackage{listings}

\usepackage{fixltx2e}

\usepackage[T1]{fontenc}

\usepackage[T1]{fontenc}

\usepackage{color}
 
\definecolor{codegreen}{rgb}{0,0.6,0}
\definecolor{codegray}{rgb}{0.5,0.5,0.5}
\definecolor{codepurple}{rgb}{0.58,0,0.82}
\definecolor{backcolour}{rgb}{0.95,0.95,0.92}
 
\lstdefinestyle{mystyle}{
    backgroundcolor=\color{backcolour},   
    commentstyle=\color{codegreen},
    keywordstyle=\color{magenta},
    numberstyle=\tiny\color{codegray},
    stringstyle=\color{codepurple},
    basicstyle=\footnotesize,
    breakatwhitespace=false,         
    breaklines=true,                 
    captionpos=b,                    
    keepspaces=true,                 
    numbers=left,                    
    numbersep=5pt,                  
    showspaces=false,                
    showstringspaces=false,
    showtabs=false,                  
    tabsize=2
}
 
\lstset{style=mystyle}

%\sectionfont{\fontsize{10}{10}\selectfont}


\begin{document}


\author{
		Harman Kumar\\
		2013CS10224		
		}

\title{Design Document}
\maketitle



% \begin{center}

% Problem Statement : \href { http://www.cse.iitd.ac.in/~prathmesh/ta/COP290/ass1.html }{Assignment Link}

% \end{center}


\section{Overall Design}

\begin{flushleft}
The whole project is divided into the following parts:

\begin{enumerate}

\item Implementing the binomial heap and fibonacci heap data structures.

\item Using the above mentioned data structures, prims MST algorithm and the fredman-tarjan MST algorithm were implemented.

\item Generating random graphs on which the the running times of the above mentioned MST algorithms was tested. 

\item A script for the automated testing of the algorithms on the generated graphs and timing them.  

\end{enumerate} 

\end{flushleft} 



\section{Random Graph Generation}
\begin{flushleft}
	The random graph generator 
	
	Some features of the random graph generator are as follows:
	
\begin{enumerate}
	\item The generated graph is guaranteed to be connected. This is done by generating a tree and adding edges to it, until a graph of the desired properties is not formed.

	\item The number of vertices and desired number of edges were taken as user parameters and the graph is generated accordingly.
	
	\item The approach of adding vertices to the tree is as follows:\\
	\vspace{10px}
Pick two random vertices:	
	\begin{enumerate}
	\item If they are connected, do nothing.
	\item If not, add an edge with a randomly generated weight between them.
	\end{enumerate}
Do the above till either the desired number of edges is reached \textbf{OR} there is a timeout condition.	
\end{enumerate}
	After the graph 
\end{flushleft}

\section{Prims Algorithm and The Binary Heap}
\begin{flushleft}
	The implementation of the binary heap provided by the boost heap library was used.\\
	The graph is read from a text file and stored in the form of an adjacency list. Prims algorithm is run on the graph and the weight of the MST is returned.
	
	  
\end{flushleft} 

\section{Fredman-Tarjan Algorithm and The Fibonacci Heap}
\begin{flushleft}
	In order to implement the Fredman-Tarjan algorithm, the fibonacci heap data structure was implemented.\\
	\lstinputlisting[language=c++]{Fib.h}	
	\vspace{10px}	
	In order to maintain disjoint sets of vertices, the Union-Find data structure is implemented as follows:
	\lstinputlisting[language=c++]{UF.h}

	The graph is read from a text file and stored in the form of an adjacency list. The Fredman-Tarjan algorithm is applied on the graph and the weight of the MST is given as output.
	
	


\end{flushleft} 

\section{Automated Testing}
\begin{flushleft}
	This is done by a python script (Automated\_Tester.py).
	This script Generates a random graph, and computes the running times that the two algorithms take to compute the MST corresponding to that graph.\\
	The timing of the algorithms was done inside the C++ code itself by using the <time.h> library. The time was that taken by the algorithm to compute the MST of the graph, exclusive of any time spent on file I/O etc.\\
	While compiling the programs, the \textbf{-o0} flag was used in order to turn off all compiler optimizations, also disk caching was disabled, in order to get a better estimate of how the programs behave.
	
	   
\end{flushleft}

\section{Acknowledgements}
\begin{flushleft}
\begin{itemize}
\item Boost C++ libraries, for the boost::binary\_heap, an implementation of binary heap (used in the implementation of prims algorithm).
\item Introduction to Algorithms by CLRS, for the pseudocode for fibonacci heap, that was translated into C++ code. 
\item g++ documentation and windows caching documentation for turning off all levels of compiler optimizations and caching.
\end{itemize}
\end{flushleft}


\end{document}