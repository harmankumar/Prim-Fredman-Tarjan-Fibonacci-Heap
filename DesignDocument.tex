\documentclass[]{article}

\usepackage[parfill]{parskip}

\usepackage{hyperref}

\usepackage[T1]{fontenc}

\usepackage[T1]{fontenc}

%\sectionfont{\fontsize{10}{10}\selectfont}


\begin{document}


\author{
		Harman Kumar\\
		2013CS10224		
		}

\title{Assignment 1}
\maketitle



% \begin{center}

% Problem Statement : \href { http://www.cse.iitd.ac.in/~prathmesh/ta/COP290/ass1.html }{Assignment Link}

% \end{center}


\section{Overall Design}

\begin{flushleft}
The whole project is divided into the following parts:

\begin{enumerate}

\item Implementing the binomial heap and fibonacci heap data structures.

\item Using the above mentioned data structures, prims MST algorithm and the fredman-tarjan MST algorithm were implemented.

\item Generating random graphs on which the the running times of the above mentioned MST algorithms was tested. 

\item A script for the automated testing of the algorithms on the generated graphs and timing them.  

\end{enumerate} 

\end{flushleft} 



\section{Random Graph Generation}
\begin{flushleft}
	The random graph generator 
	
	Some features of the random graph generator are as follows:
	
\begin{enumerate}
	\item The generated graph had to be connected.

	\item The density (a measure of the number of edges) of the graph was a parameter of the graph generator.
	
	\item 
\end{enumerate}
	
\end{flushleft}

\section{Prims Algorithm and The Binary Heap}
\begin{flushleft}
	The implementation of the binary heap provided by the boost heap library was used.
	  
\end{flushleft} 

\section{Fredman-Tarjan Algorithm and The Fibonacci Heap}
\begin{flushleft}
	In order to implement the Fredman-Tarjan algorithm, the fibonacci heap data structure was implemented.
	
	

	The Fredman-Tarjan algorithm is implemented in C++, the fibonacci heap data structure is provided in the boost library, fibonacci\_heap.hpp.  
\end{flushleft} 

\section{Automated Testing}
\begin{flushleft}
	This is done by a python script (Automated\_Tester.py).
	This script Generates a random graph, and saves the running times that the two algorithms took to compute it's MST.   
\end{flushleft}




\end{document}